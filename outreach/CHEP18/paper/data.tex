\section{Dataset}
At the work we focused on electrons interactions inside the electromagnetic calorimeter at the LHCb. In particular, the calorimeter used in this study employs "shashlik" technology of alternating scintillating tiles and lead plates. It consists of 5 $\times$ 5 blocks of size 12 cm $\times$ 12 cm, the cell granularity corresponds to each block is 5 $\times$ 5 of size 2 cm $\times$ 2 cm. There are 66 layers in ECAL -- 2 mm absorber and 4 mm scintillator. In fact, the shower appears in 3d, but we summarized allocated energies in each layer per cell. This procedure does not obstruct physics analysis and does not inhibit the shower shape. Thus, this information can be represented as 30 $\times$ 30 images $Y$ with the corresponding parameters $(p_x,~ p_y,~ p_z,~ x,~ y)$. Such image example is presented in~\cref{fig:real-imgs}.

The training data set is created as follows. \geant is utilized to generate particles and simulate their interaction with the calorimeter using the \texttt{Ftfp\_Bert} physics library based on the \texttt{Fritiof}  and \texttt{Bertini} intra-nuclear cascade models with the standard electromagnetic physics package. So information about every event includes the parameters and 30 $\times$ 30 matrix of energies deposited in scintillator for every cell tower $Y$. Size of the training dataset is 50 000 events, and we have 10 000 events at the test dataset.

\todo{Fedor. F: moved here from ``method''} 