\section{Conclusion and outlook}\label{conclusion}

The research proves that Generative Adversarial Networks are a good candidate for fast simulation of high granularity detectors typically studied for the next generation accelerators. We have successfully generated images of shower energy deposition with a condition on the particle parameters, such as the momentum and the coordinates, using modern generative deep neural network techniques such as Wasserstein GAN with gradient penalty.

Future work will be focused on improving reproduction of second-order cluster characteristics, such as variations and long-range correlations between different cells.

The research leading to these results has received funding from the Russian Science Foundation under agreement No 19-71-30020. 
