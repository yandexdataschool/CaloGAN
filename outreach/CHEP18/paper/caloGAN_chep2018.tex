\documentclass{webofc}
\usepackage[utf8]{inputenc}
\usepackage{graphicx}
\usepackage[varg]{txfonts}
\usepackage{subcaption}
\usepackage{appendix}
\usepackage{cleveref}
\usepackage{amsmath}
\usepackage{amssymb}
\usepackage{float}
\usepackage{bm}
\usepackage{xcolor}

% \newcommand{\onedot}{\futurelet\@let@token\@onedot}

\newcommand{\onedot}{.\\}

\newcommand{\eg}{\emph{e.g}\onedot} \newcommand{\Eg}{\emph{E.g}\onedot}
\newcommand{\ie}{\emph{i.e}\onedot} \newcommand{\Ie}{\emph{I.e}\onedot}
\newcommand{\cf}{\emph{c.f}\onedot} \newcommand{\Cf}{\emph{C.f}\onedot}
\newcommand{\etc}{\emph{etc}\onedot} \newcommand{\vs}{\emph{vs}\onedot}
\newcommand{\wrt}{w.r.t\onedot} \def\dof{d.o.f\onedot}
\newcommand{\etal}{\emph{et al}\onedot}


\newcommand{\vx}{\ensuremath{\mathbf{x}}}
\newcommand{\vz}{\ensuremath{\mathbf{z}}}
\newcommand{\vy}{\ensuremath{\mathbf{y}}}

\newcommand{\f}{f}

\newcommand{\dsep}{\ensuremath{\;\|\;}}

\newcommand{\pdata}{\ensuremath{p_{\text{data}}}}
% \newcommand{\pfake}{\ensuremath{p_{\text{fake}}}}
\newcommand{\pfake}{\ensuremath{ p_{G} }}

\newcommand{\todo}[1]{{\color{blue}TODO: #1}}

\newcommand{\E}{\ensuremath{\mathbb{E}}}

\newcommand{\geant}{\texttt{Geant4} }


\begin{document}

\title{Generative Models for Fast Calorimeter Simulation}
\subtitle{LHCb case}

\author{
 \firstname{Viktoria} \lastname{Chekalina} \inst{1,2}\fnsep\thanks{\email{sayankotor1@gmail.com}}
\and
    \firstname{Elena} \lastname{Orlova} \inst{3}\fnsep\thanks{\email{egorlova68@gmail.com}}
\and
    \firstname{Fedor} \lastname{Ratnikov} \inst{1,2}
\and
     \firstname{Dmitry} \lastname{Ulyanov} \inst{3}
\and
     \firstname{Andrey} \lastname{Ustyuzhanin} \inst{1,2}
\and
     \firstname{Egor} \lastname{Zakharov} \inst{3}\fnsep\thanks{\email{eo.zakharov@gmail.com}}
}

\institute{
NRU Higher School of Economics, Moscow, Russia
\and
Yandex School of Data Analysis, Moscow, Russia  
\and
Skolkovo Institute of Science and Technology, Moscow, Russia
}
        
\abstract{Simulation is one of the key components in high energy physics. Historically it relies on the Monte Carlo methods which require a tremendous amount of computation resources. These methods may have difficulties with the expected High Luminosity Large Hadron Collider (HL LHC) need, so the experiment is in urgent need of new fast simulation techniques. We introduce a new Deep Learning framework based on Generative Adversarial Networks which can be faster than traditional simulation methods by 5 order of magnitude with reasonable simulation accuracy. This approach will allow physicists to produce a big enough amount of simulated data needed by the next HL LHC experiments using limited computing resources.}


\maketitle


\section{Introduction}

Simulation plays an important role in particle and nuclear physics. It is widely used in detecor design and to compare experimental data to theoretical models. Traditionally, simulation relies on \textit{Monte Carlo methods} and requires significant computational resources. Such methods does not not scale to meet the growing demands resulting from large quantities of data expected during High Luminosity Large Hadron Collider (HL LHC) runs. The detailed simulation of particle collisions and interactions as captured by detectors at the LHC using a well known simulation software \geant annually requires billions of CPU hours constituting more than half of the LHC experiments' computing resources \cite{bozzi2014, flynn2015computing}. More specifically, the detailed simulation of particle showers in calorimeters is the most computationally demanding step.
 
A line of simulation methods exploit the idea of reusing previously calculated or measured physical quantities have been developed to reduce the computation time \cite{grindhammer2000parameterized,atlas2010simulation}. These approach suffers from being specific to an individual experiment and, despite being faster than the full simulation, they still take relatively long to apply. Thus, the particle physics community is in need of new faster simulation methods to model experiments. 
    
One of the possible approaches to simulate the calorimeter response is using the \textit{deep learning} techniques. In particular, a recent work, CaloGAN \cite{paganini2017calogan}, provided an evidence that \textit{Generative Adversarial Networks} can be used to efficiently simulate particle showers. While they achieved over $100,000 \times$ speed-up over \geant, their setup was quite simple as they parametrized an input particle only by its energy. Moreover, the quality of their simulation is unsatisfactory \todo{Fedor, is it true?}. 

In this work we build a model upon \text{Wasserstein Generative Advesarial Networks} and show its superior performance over CaloGAN. We also evaluate our model in a more complex scenario, when a particle is described by $5$ parameters: 3d momentum $(p_x,~ p_y,~ p_z)$ and 2d coordinate $(x,~ y)$. Our method for high-fidelity fast simulation of particle showers in the specific LHCb calorimeter aims to replace the existing Monte Carlo based methods and achieve a significant speed-up factor.
 

% Our contirubtion is three fold: 

% \begin{itemize}
%     \item We build and extensively evaluate a simupropose a  
%     \item 
%     \item Dataset? 
% \end{itemize}




\section{Related work: GANs basics and GANs in HEP}
Generative models are of great interest in deep learning. With these models, one can approximate a very complex distribution defined as a set of samples. 
For example, such models can be utilized to generate a face image of a non-existing person or to continue a video sequence given several initial frames. 
In this section, we give a brief overview of the most popular generative model in computer vision — Generative Adversarial Networks (GANs),
 its strong and weak sides and different modifications to alleviate its weaknesses. Then, we review and analyse current approaches for applying GANs to the simulation of calorimeters in High energy physics.


\section{Dataset}
At the work we focused on electrons interactions inside the electromagnetic calorimeter at the LHCb. In particular, the calorimeter used in this study employs "shashlik" technology of alternating scintillating tiles and lead plates. It consists of 5 $\times$ 5 blocks of size 12 cm $\times$ 12 cm, the cell granularity corresponds to each block is 5 $\times$ 5 of size 2 cm $\times$ 2 cm. There are 66 layers in ECAL -- 2 mm absorber and 4 mm scintillator. In fact, the shower appears in 3d, but we summarized allocated energies in each layer per cell. This procedure does not obstruct physics analysis and does not inhibit the shower shape. Thus, this information can be represented as 30 $\times$ 30 images $Y$ with the corresponding parameters $(p_x,~ p_y,~ p_z,~ x,~ y)$. Such image example is presented in~\cref{fig:real-imgs}.

The training data set is created as follows. \geant is utilized to generate particles and simulate their interaction with the calorimeter using the \texttt{Ftfp\_Bert} physics library based on the \texttt{Fritiof}  and \texttt{Bertini} intra-nuclear cascade models with the standard electromagnetic physics package. So information about every event includes the parameters and 30 $\times$ 30 matrix of energies deposited in scintillator for every cell tower $Y$. Size of the training dataset is 50 000 events, and we have 10 000 events at the test dataset.

\todo{Fedor. F: moved here from ``method''} 
\subsection{Background: from GAN to conditional WGAN}

Generative Adversarial Networks (GANs) were originally presented by I.~Goodfellow~\etal in 2014 \cite{goodfellow2014generative} and quickly became a state-of-the-art technique in areas such as image generation \cite{radford2015unsupervised}, with huge number of extensions \cite{IsolaZZE16,CycleGAN2017,wang2018video}.

In the GAN framework the aim is to learn a mapping $G$, usually called \textit{generator}, to warp an easy-to-draw distribution $p(\vz)$ (e.g. $p(\vz) = \mathcal{N}(0, I)$) into a target distribution $\pdata(\vx)$ to facilitate sampling from $\pdata(\vx)$. When $G$ is learned $G \equiv G^*$, sampling from the target distribution $\pdata(\vx)$ is done by first drawing a sample from the distribution $p(\vz)$ and then feeding the sample into the generator: $G^*(\vz) \sim \pdata$, where $\vz \sim p(\vz)$. For such sampling procedure, the time needed to draw a sample from $\pdata(\vx)$ is approximately equal to the time needed to evaluate the function $G$ in a point.  

The generator is learned by using a feedback from an external classifier (usually called \textit{discriminator}), which tries to find discrepancy between the target distribution $\pdata(\vx)$ and fake distribution $\pfake(\vx)$ defined by samples from the generator $G(\vz) \sim \pfake(\vx),\, \vz \sim p(\vz)$. %The process in summarised in~\cref{fig:GANs}.

% \begin{figure}
% \centering
% \includegraphics[width=0.3\linewidth]{figures/gan_pic.pdf}
% \caption{Generative Adversarial Networks for digit generation. The generator $G$ transforms the noise vector $\vz \sim p(z)$ to an image of a digit and the discriminator $D$ classifies inputs as real digits or fake digits from generator. Generator and discriminator are trained in an adversarial manner: the task of $G$ is to make it impossible for $D$ to distinguish between the real and fake digits as in this case $G$ reproduces the data distribution $\pdata$.}\label{fig:GANs}
% \end{figure}


More formally, generator $G$ and discriminator $D$ play the following zero sum game: 
\begin{equation}\label{eq:gan}
\min_G \max_D \E_{\vx \sim \pdata(\vx)} [\log D(\vx)] + \E_{\vx \sim \pfake(\vx)} [\log(1 - D(\vx))]\, ,
\end{equation} 
where $D(G(\vz))$ is the output of the discriminator specifying the probability of its input to come from the target distribution $\pdata$.

In practice the mappings $G$ and $D$ are parametrized by deep neural networks and the objective~\cref{eq:gan} is optimized using alternating gradient descent. For a fixed generator, the discriminator minimizes binary cross-entropy in a binary classification problem (samples from $\pdata$ versus samples from $\pfake$). For a fixed discriminator, generator is updated to make its samples to be misclassified by discriminator, thus moving fake distribution closer to the target distribution.   

It is possible to show, that for a fixed generator the optimal value for the inner optimization can be written analytically: 
\begin{equation}\label{eq:js}
\max_D \E_{\vx \sim \pdata(\vx)} [\log D(\vx)] + \E_{\vx \sim \pfake(\vx)} [\log(1 - D(\vx))] = \text{JS}( \pdata \dsep \pfake)\, , 
\end{equation} 
where $\text{JS}$ is Jensen-Shannon divergence. In fact, for a fixed generator (hence fixed fake distribution), the discriminator computes the divergence between the target distribution $\pdata$ and the fake distribution $\pfake$. When the divergence is computed, the generator aims to update fake distribution to make this divergence lower: $\min_G \text{JS}( \pdata \dsep \pfake )$. While Jensen-Shannon divergence naturally arises from the original game~\cref{eq:gan}, any divergence or distance $\mathcal{D}$ can be used instead: $\min_G \mathcal{D}( \pdata \dsep \pfake )$.

A recent work~\cite{arjovsky2017wasserstein} proposed to use \textit{Wasserstein distance} instead of Jensen-Shannon divergence proving its better behavior:
\begin{equation}\label{eq:wasserstein_metric}
W(\pdata \dsep \pfake ) = \max_{\f \in \mathcal{F}} \E_{\vx \sim \pdata(\vx)}[\f(\vx)] - \E_{\vx \sim \pfake} [\f(\vx)]
\end{equation}
where $\mathcal{F}$ is a set of 1-Lipshitz functions. Using Wasserstein distance instead of Jensen-Shannon divergence in a GAN objective leads to the Wasserstein GAN (WGAN) objective: 
\begin{equation}\label{eq:wgan}
\min_G \max_{\f \in \mathcal{F}} \E_{\vx \sim \pdata(\vx)}[\f(\vx)] - \E_{\vx \sim \pfake(\vx)} [\f(\vx)]\, .
\end{equation}

It is highly non-trivial to search over the set of 1-Lipshitz functions and several ways has been proposed in order to force this constraint \cite{arjovsky2017wasserstein,gulrajani2017improved}. \cite{gulrajani2017improved} proved, that the set of optimal functions for~\cref{eq:wgan} contains such function, that norm of it's gradient in any point equals one. In practice this result motivates an additional loss added to the objective~\cref{eq:wgan} with a weight $\lambda$ while the hard constraint on the function $\f$ to belong to the set $\mathcal{F}$ is removed and $\f$ is searched over all possible functions:    
\begin{equation}\label{gpwgan-loss}
% \begin{gathered}
\min_G \max_\f \E_{\vx \sim \pdata(\vx)}  \f(\vx) - \E_{\vx \sim \pfake(\vx)} \f(\vx) + 
\lambda \E_{\vx \sim \pfake} \big(\|\nabla_{\tilde{\vx}} D(\tilde{\vx})\|_2 - 1\big)^2 .
% \end{gathered}
\end{equation}

WGAN can be easily adapted to model a conditional distribution $\pdata(\vx | \vy)$. The generator is modified to take the condition along with the sample $\vz$ so the fake distribution is now defined as $G(\vz, \vy) \sim \pfake(\vx | \vy),\, \vz \sim p(\vz)$ and the game is 
\begin{equation}\label{CWGAN}
% \begin{gathered}
\min_G \max_\f \E_{\vy \sim p(\vy)} \Big[ \E_{\vx \sim \pdata(\vx | \vy)}  \f(\vx) - \E_{\vx \sim \pfake(\vx | \vy)} \f(\vx) + 
\lambda \E_{\vx \sim \pfake(\vx | \vy)} \big(\|\nabla_{\tilde{\vx}} D(\tilde{\vx})\|_2 - 1\big)^2 \Big]\,.
% \end{gathered}
\end{equation}


% \subsubsection{Wasserstein GAN}
% There are various modifications of GANs for struggling with some typical problems and for improving the training procedure. A common GANs issue is so--called mode collapse when $p_\text{model} (\vx)$ fails to capture a multimodal nature of $\pdata(\vx)$ and in extreme cases all the generated samples might be identical, in more involved architectures such as Waserstain GAN \cite{arjovsky2017wasserstein} the discriminator loss is argued to be consistent with the image quality. The main idea is to apply the Wasserstein-1 distance in order to compare $\pdata$ and $p_{\text{model}}$.

% Let $\mathbb{X}$ be a compact metric set and let $\mathbb{B}$ denote the set of all the Borel subsets of $\mathbb{X}$. Let $Prob(\mathbb{X})$ denote the space of probability measures defined on $\mathbb{X}$. The \emph{Earth-Mover} or \emph{Wasserstein-1} distance between two distributions $\mathbb{P}_r, \mathbb{P}_g \in Prob(\mathbb{X})$ is defined in the following way:

% \begin{equation}\label{wasserstein_metric}
% W(\mathbb{P}_r, \mathbb{P}_g) = \inf_{\gamma \in \Pi(\mathbb{P}_r, \mathbb{P}_g)} \mathbb{E}_{(x, y) \sim \gamma} \big{[}\|\vx-\textbf{y}\|\big{]},
% \end{equation}

% where $\Pi(\mathbb{P}_r, \mathbb{P}_g)$ is the set of all joint distributions $\gamma(x, y)$ whose marginals are respectively $\mathbb{P}_r$ and $\mathbb{P}_g$. $\gamma(x, y)$ denotes how much “mass” must be transported from $x$ to $y$ in order to transform the distributions $\mathbb{P}_r$ into the distribution $\mathbb{P}_g$. Thus, the Wasserstein distance can be defined as the minimum cost of transporting mass in order to transform the distribution $\mathbb{P}_r$ into the distribution $\mathbb{P}_q$. 

% By applying to this distance the \emph{Kullback--Leibler divergence}, the \emph{Jensen-Shannon divergence} and the Kantorovich--Rubinstein duality and also parameterization of family of functions $\{f_w\}_{w \in \mathcal{W}}$ that are all K-Lipschitz for some K we could consider solving the problem (for more detailed information see the paper \cite{arjovsky2017wasserstein})

% \begin{equation}\label{optim}
% \max_{w \in \mathcal{W}} \mathbb{E}_{x \sim P_r}[f_w(x)] - \mathbb{E}_{z\sim p(z)} [f_w(g_\theta(z))]
% \end{equation}

% and if the supremum in \eqref{optim} is attained for some $w \in \mathcal{W}$, this process would lead to a calculation of $W(\mathbb{P}_r, \mathbb{P}_\theta)$ up to a multiplicative constant.
% So, the WGAN value function is
% \begin{equation}\label{wgan_loss}
% \min_G \max_{D \in \mathcal{D}}  \mathbb{E}_{\vx \sim \mathbb{P}_r}  [D(\vx)] - \mathbb{E}_{\tilde{\vx} \sim \mathbb{P}_g} [D(\tilde{\vx})],
% \end{equation}
% where $\mathcal{D}$ is the set of 1-Lipschitz functions and $\mathbb{P}_g$ is  the model distribution defined by $\tilde{\vx} = G(\vz), ~\vz \sim p(\vz).$

% In general, the training procedure of GANs is known to be difficult and presents such issues as mode collapse, and WGAN often helps to overcome this problem.

% The Wasserstein GAN value function makes optimization process easier because the discriminator's gradient with respect to its input is better behaved than its GAN counterpart. Also, it was noticed that the WGAN value function tend to correlate with the original data quality which is not always the case for ordinary GANs.


% \subsubsection{Wasserstain GAN with gradient penalty}
% An alternative way to enforce the Lipschitz constraint was introduced in \cite{gulrajani2017improved}. The authors consider directly constraining the gradient norm of the discriminator's output with respect to its input because a differentiable function is 1-Lipschtiz if and only if it has gradients with norm at most 1 everywhere. To come over tractability issues a soft version of the constraint with a penalty on the gradient norm for random samples $\tilde{\vx} \sim \mathbb{P}_{\tilde{\vx}}$ was suggested. Therefore, a new obtained objective is 
% \begin{equation} \label{gpwgan-loss}
% \begin{gathered}
% \mathcal{L}(\bm{\theta}) =
% \underbrace{ \underset{\tilde{\vx} \sim \mathbb{P}_g}{\mathbb{E}}  \Big[D(\tilde{\vx})\Big] - \underset{\vx \sim \mathbb{P}_r}{\mathbb{E}} \Big[D(\vx)\Big]}_{\text{original loss}} + 
% \underbrace{ \lambda \underset{\vx \sim \mathbb{P}_g}{\mathbb{E}} \Big[\big(\|\nabla_{\tilde{\vx}} D(\tilde{\vx})\|_2 - 1\big)^2 \Big]}_{\text{gradient penalty}}.
% \end{gathered}
% \end{equation}

% This approach demonstrates strong modeling performance and stability across a variety of architectures.  Now it is a state-of-the-art technique in GANs. 
% How to calculate this loss in practice is described on~\cref{sec:training_strategy}.

\subsection{GANs in high energy physics}
A  systematic study on the application of deep learning to simulation of calorimeters for particle physics has been carried out by Paganini et al. in 2017 \cite{paganini2017calogan} and has resulted in the CaloGAN package. The authors aim to speed up particle simulation in a 3-layer heterogeneous calorimeter  using GANs framework and achieve $\sim \times 10^5 $ speedup. They used an existing state-of-the-art (but slow) simulation engine \geant to create a training dataset. They simulated positrons, photons and charged pions with various energies sampled from a flat distribution between 1 GeV and 100 GeV. All incident particles in this study have initial momentum perpendicular to the face of the calorimeter. The shower in the first layer is represented as a $3 \times 96$ pixel image, the middle layer as a $12 \times 12$ pixel image, and the last layer as a $12 \times 6$ pixel image. 

Their design of the generator network is based on DCGAN structure \cite{radford2015unsupervised} with some convolutional layers replaced by locally-connected layers \cite{taigman2014deepface}. The idea of locally connected layers is based on the fact that every pixel position gets its own filter while an ordinary convolutional layer is applied over the whole image, independently of location. Extension of this method to particle physics simulation has been described in the previous work of the authors where the resulting type of neural network was called LAGAN \cite{de2017learning}. A special section in the paper is devoted to the evaluation of the quality of the CaloGAN produced images where  sparsity level,  energy per layer or total energy are used as measures of the performance of the model. 

The obtained results demonstrate a prospect of application of GANs for the particle showers generation and replacing of the Monte Carlo methods with the proposed approach. The CaloGAN approach yields sizeable simulation-time speed ups compared to \geant. 

%In fact, the CaloGan model is based on DCGAN with the described tricks. However, GANs tend to suffer from mode collapse. Therefore, the CaloGan architecture cannot be applied for all datasets, because %here is a high probability of mode collapse appearance and it is a limitation of this work.

\section{Our model} \label{sec:model}
Our idea is to treat simulations as a black-box and replace the traditional Monte Carlo simulation with a method based on Generative Adversarial Networks. WGANs with gradient penalty are considered to be state-of-the-art technique for image producing, so we decide to implement a tool based on this particular approach. For it to be useful in realistic physics applications such a system needs to be able to accept requests for the generation of showers originating from an incoming particle parameters such as 3d momentum and 2d coordinate. We introduce an auxiliary task of energy reconstruction to condition on these parameters $p_x$, $p_y$, $p_z$ and $x$, $y$.

\subsection{Model architecture}



We need to generate a specific calorimeter response to a particle with some parameters. It means that a model is required to be conditional.
% sampling not just from $p(\textbf{y}),$ but from $p(\textbf{y}|\vx),$ so, 
Firstly, we describe a generator and discriminator architecture. The generator maps from an input (a 512 $\times$ 1 vector sampled from the Gaussian distribution and the particle parameters) to an 30 $\times$ 30 image $\hat{\textbf{y}}$ using deconvolutional layers (in fact, it is an upsampling procedure and convolutions) which are arranged as follows. We concatenate the noise vector and the parameters $(p_x,~ p_y,~ p_z,~ x,~ y)$, after that we add a fully connected layer with reshaping and obtain 256 $\times$ 4 $\times$ 4 output. After a sequence of 2d deconvolutions we get outputs of size  128 $\times$ 8 $\times$ 8, 64 $\times$ 15 $\times$ 16 and 32 $\times$ 32 $\times$ 32  with ReLu activation functions. After this procedure we crop the last output to obtain the image of desired size 30 $\times$ 30.

As for the discriminator, it takes a batch of images as input (all images in the batch are real or generated by $G$) and returns the score $D(\textbf{y})$ or $D(\hat{\textbf{y}})$ as it is described in \cite{arjovsky2017wasserstein}. Discriminator architecture is simply the reversed generator architecture (i.e. sizes of layers go in the opposite order). It implies that we have 30 $\times$ 30 matrix as input, then we obtain layers outputs of size 32 $\times$ 32 $\times$ 32, 64 $\times$ 15 $\times$ 16, 128 $\times$ 8 $\times$  8, after that reshaping leads to 256 $\times$ 4 $\times$ 4, and by applying LeakyRelu activation function we get the final score. The model scheme is presented in~\cref{fig:model}.

\begin{figure}
\centering
\includegraphics[width=1\textwidth]{figures/model_architecture.pdf}
\caption{Model architecture. Pre-trained regressor for the particle parameters prediction makes our model conditional. Thanks to building up the information from the pre-trained regressor into the discriminator gradient we learn $G$ to produce a specific calorimeter response.}\label{fig:model}
\end{figure}

How to train WGAN with gradient penalty in conditional manner is described in the following section.

\subsection{Training strategy} \label{sec:training_strategy}
Due to the nature of WGAN loss, conditioning on the continuous value is a non-trivial task. To overcome this issue we suggest to embed a pre-trained regressor in our model. We train a neural network to predict the particle parameters by the calorimeter response. As for architecture, it has the same one as the discriminator but with a perceptual loss described in \cite{johnson2016perceptual} because, as we figured out, it works better rather than standard MSE. By building up the information from the pre-trained regressor into the discriminator gradient, we obtain the conditional model because we train the generator and the discriminator together. Now the discriminator makes the generator produce a specific calorimeter response.

Matrices from our dataset are pretty sparse because almost all information is located in central cells (see~\cref{fig:real-imgs}). To make optimization process easier we apply a box--cox transformation. This mapping helps to smooth the data that makes the optimization process more stable.
Results obtained with the described model are presented in the following section.

PyTorch \cite{pytorch} library was used to carry out model training and experiments.

\begin{figure}
\begin{center}
\includegraphics[width=0.5\textwidth]{figures/mean_cluster.pdf}
\caption{energy deposition in different cells of used 30$\times $30 setup for \geant simulated events averaged over all events in the used dataset. \label{fig:real-imgs}}
\end{center}
\end{figure}
\section{Experiments}

\begin{figure}
\captionsetup[subfigure]{justification=centering}
  \centering
  \begin{subfigure}{0.24\textwidth}
    \centering
    \includegraphics[width=1\textwidth]{figures/1_real.png}
    % \caption{\\$E = 63.7~\text{GeV}$ \\ $\frac{p_x}{p_z}=0.005$ \\ $\frac{p_y}{p_z}=0.154$}\label{fig:real-imgs-1}
  \end{subfigure}
  \begin{subfigure}{0.24\textwidth}
    \centering
    \includegraphics[width=1\textwidth]{figures/2_real.png}
    % \caption{\\$E = 6.5~\text{GeV}$ \\  $\frac{p_x}{p_z}=0.0046$ \\$\frac{p_y}{p_z}=0.108$}\label{fig:real-imgs-2}
  \end{subfigure}
    \begin{subfigure}{0.24\textwidth}
    \centering
    \includegraphics[width=1\textwidth]{figures/3_real.png}
    % \caption{\\$E = 15.6~\text{GeV}$ \\ $\frac{p_x}{p_z}=0.196$ \\ $\frac{p_y}{p_z}=-0.036$}\label{fig:real-imgs-3}
  \end{subfigure}
  \begin{subfigure}{0.24\textwidth}
    \centering
    \includegraphics[width=1\textwidth]{figures/4_real.png}
    % \caption{\\$E = 15.6~\text{GeV}$ \\  $\frac{p_x}{p_z}=-0.019$ \\ $\frac{p_y}{p_z}=0.181$}\label{fig:real-imgs-4}
  \end{subfigure}\\
   \begin{subfigure}{0.24\textwidth}
    \centering
    \includegraphics[width=1\textwidth]{figures/1_gen.png}
    \caption{\\$E_0 = 63.7~\text{GeV}$ }%\\ $\frac{p_x}{p_z}=0.005$ ,  $\frac{p_y}{p_z}=0.154$}}%\label{fig:gen-imgs-1}
  \end{subfigure}
  \begin{subfigure}{0.24\textwidth}
    \centering
    \includegraphics[width=1\textwidth]{figures/2_gen.png}
    \caption{\\$E_0 = 6.5~\text{GeV}$ }% \\  $\frac{p_x}{p_z}=0.046$ , $\frac{p_y}{p_z}=0.108$}}%\label{fig:gen-imgs-2}
  \end{subfigure}
    \begin{subfigure}{0.24\textwidth}
    \centering
    \includegraphics[width=1\textwidth]{figures/3_gen.png}
    \caption{\\$E_0 = 15.6~\text{GeV}$ }% \\ $\frac{p_x}{p_z}=0.196$ ,  $\frac{p_y}{p_z}=-0.036$}}%\label{fig:gen-imgs-3}
  \end{subfigure}
  \begin{subfigure}{0.24\textwidth}
    \centering
    \includegraphics[width=1\textwidth]{figures/4_gen.png}
    \caption{\\$E_0 = 15.9~\text{GeV}$ }% \\  $\frac{p_x}{p_z}=-0.019$ ,  $\frac{p_y}{p_z}=0.181$}}%\label{fig:gen-imgs-4}
  \end{subfigure}
 
  \caption{Showers generated with \geant (first row) and the showers,
    simulated with our model (second row) for three different sets of
    input parameters. Color represents $log_{10}(\frac{E}{MeV})$ for every cell}
  \label{fig:geant_vs_ours}
\end{figure}

We start with comparing original clusters, produced by full \geant
simulation and clusters generated by the trained model for the same
 parameters of the incident particles: the same energy, the same direction,
 and the same position on the calorimeter face. Corresponding images
 for four arbitrary parameter sets are presented
 in~\cref{fig:geant_vs_ours}. These images demonstrate very good
 visual similarity between simulated and generated clusters.


\begin{figure}
  \centering
  \begin{subfigure}[t]{0.3\textwidth}
    \centering
    \includegraphics[width=1\textwidth]{figures/width_trans.pdf}
    \caption{The transverse width of real and generated clusters}
  \end{subfigure}\hspace{0.2\textwidth}
 \begin{subfigure}[t]{0.3\textwidth}
    \centering
    \includegraphics[width=1\textwidth]{figures/width_long.pdf}
    \caption{The longitudinal width of real and generated clusters}
  \end{subfigure}
  \begin{subfigure}[t]{0.3\textwidth}
    \centering
    \includegraphics[width=1\textwidth]{figures/deltaX.pdf}
    \caption{$\Delta X$ between cluster center of mass and the true particle coordinate}
  \end{subfigure}\hspace{0.2\textwidth}
  \begin{subfigure}[t]{0.3\textwidth}
    \centering
    \includegraphics[width=1\textwidth]{figures/sparsity.pdf}
    \caption{The sparsity of real and generated clusters}
  \end{subfigure}
  \begin{subfigure}[t]{0.3\textwidth}
    \centering
    \includegraphics[width=1\textwidth]{figures/transverseAsymmetry.pdf}
    \caption{The transverse asymmetry of real and generated clusters}
  \end{subfigure}\hspace{0.2\textwidth}
  \begin{subfigure}[t]{0.3\textwidth}
    \centering
    \includegraphics[width=1\textwidth]{figures/longAsymmetry.pdf}
    \caption{The longitudinal asymmetry of real and generated clusters}
  \end{subfigure}
  \caption{Generated images quality evaluation including described physical characteristics.}\label{fig:quality}  
\end{figure}

Then we continue with quantitative evaluation of the proposed simulation
method. While generic evaluation methods for generative models exist,
here we base our evaluation on physics-driven similarity
metrics. These metrics are designed using the domain knowledge and the
recommendations from physicists on the evaluation of simulation
procedures. 
For this presentation we selected few cluster properties which essentially
drive cluster properties used in the reconstruction of calorimeter objects
and following physics analysis. If initial particle direction is not
perpendicular to the calorimeter face, produced cluster is elongated
in that direction. Therefore we consider separately cluster width in
the direction of the initial particle and in the transverse
direction. Spatial resolution, that is the distance between center
mass of the cluster and the initial track projection to the shower max
depth, is another important characteristics affecting physics
properties of the cluster. Cluster sparsity, that is the fraction of
cells with energies above some threshold, reflects marginal low
energy properties of the generated clusters. Finally, longitudinal and
transverse asymmetries, that are differences in energies between
forward-backward and left-right sides of the cluster, characterise
coherent energy variations.  
 These  characteristics are presented in~\cref{fig:quality}. 

The primary cluster characteristics  demonstrate good agreement with
fully simulated data. However secondary characteristics driven by
long range correlations between different cluster contributions might
be significantly improved.  




As for model performance, we trained our model for 3000 epochs which takes about 70 hours on GPU NVIDIA Tesla K80. The sampling rate is 0.07 ms per sample on GPU, 4.9 ms per sample on CPU.

\section{Conclusion and outlook}\label{conclusion}

The research proves that Generative Adversarial Networks are a good candidate for fast simulation of high granularity detectors typically studied for the next generation accelerators. We have successfully generated images of shower energy deposition with a condition on the particle parameters, such as the momentum and the coordinates, using modern generative deep neural network techniques such as Wasserstein GAN with gradient penalty.

Future work will be focused on improving reproduction of second-order cluster characteristics, such as variations and long-range correlations between different cells.

The research leading to these results has received funding from the Russian Science Foundation under agreement No 19-71-30020. 



\bibliography{caloGAN_chep2018}


\appendix

\end{document}
