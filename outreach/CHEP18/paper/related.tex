

\section{Related work}
Generative models are of great interest in deep learning. With these models one can approximate a very complex distribution defined by a set of its samples. 
For example, such models can be utilized to generate a face image of a never existed person or to continue a video sequence given several initial frames. 
In this section we give a brief overview of the most popular generative model in computer vision — Generative Adversarial Networks (GANs),
 its strong and weak sides and different modifications to alleviate its weaknesses. In~\cref{} we review analyze an article about applying GANs to the simulation problem in physics.???
